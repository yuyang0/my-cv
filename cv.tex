\documentclass[11pt,a4paper,nolmodern]{moderncv}
\usepackage[noindent,UTF8]{ctex}
\usepackage{info}
%% \usepackage{hyperref}

\title{个人简历}
%% \title{云南师范大学物电学院}
%% \myquote{自强不息,厚德载物。}{}

\begin{document}
\setmainfont{Minion Pro}
\setsansfont{Myriad Pro}

\hyphenpenalty=10000
\maketitle

\newcommand\Colorhref[3][cyan]{\href{#2}{\small\color{#1}#3}}

\section{教育背景}
\cvline{2008--2012}{云南师范大学,物理与电子信息学院,物理系}
\section{个人简介}
\cvline{}{本人,男, 1990年出生,自认为是一个求知欲很强,意志坚强,能任劳任怨的
  人,喜欢Linux,Emacs,喜欢Python,喜欢各种简洁一致的东西,热爱钻研技术,自学
  能力良好,本科虽然是物理,但是花了很多时间学习计算机,对计算机的理论知识如算
  法,数据结构,操作系统,编译原理有比较好的了解,因为兴趣关系曾经花了一段时间
  研究程序语言理论(PL),所以对各种程序语言(包括比较偏门的函数式语言如Scheme,
  ML)很熟悉, 另外请注意:本人有神经性耳聋,虽然没有全聋,但也有点严重,口头交
  流比较困难,这可能会对工作造成一定影响,但是我会尽一切可能来弥补,如果方便的
  话,请发送短信或者邮件和我联系,谢谢。}

\section{语言}
\cvlanguage{英语}{熟练}{通过全国大学英语六级考试, 能无障碍的读写英文(写一般有少量语法错误)}

\section{技能}

\subsection{开发}
\cvcomputer{程序语言}{C, Python, Php, JavaScript, Scheme}
           {Web}{HTML, CSS, jQuery, AJAX}
\cvcomputer{框架}{Django, Tornado, Scrapy,CodeIgniter}
           {数据库}{MySQL, Redis}
\cvcomputer{代码管理}{Git}
           {工具}{GitHub}
\cvcomputer{编辑器}{Emacs}
           {操作系统}{GNU/Linux(Ubuntu, Linux Mint)}


\subsection{说明}
\cvline{Python:}{对python非常熟悉,熟悉标准库,常用的第三方库,配套的开发工具(pip,
  virtualenv ipython etc),熟悉python的编码规范
  (\Colorhref{http://legacy.python.org/dev/peps/pep-0008/}{PEP8}),对python web
  开发有比较好的了解,熟悉django和tornado,从2009年接触python,至少写过10W行以
  上的python代码,python是我可以立刻用来工作的语言。}

\cvline{Linux/Shell:}{对Linux与shell比较熟悉,已经将Linux作为桌面系统使用了将近4年,用shell也写过一些脚本
  比如这个自动安装软件的 \Colorhref[red]{https://github.com/yuyang0/auto-install}{脚本}}

\cvline{web前端:}{对前端(HTML,CSS,JS)比较熟悉,但不是一个专业的前端,因为我对
  设计没有太多灵感,而且对各浏览器的兼容性了解不够,但如果只是改改模板,或者作
  为后端开发者来调整前端做好的模板,那绝对能够胜任。}


\cvline{PHP:}{对PHP比较了解,对codeignitor有一定研究,在PHP方面经验不算多,但因为
  我有其它语言的经验,所以我相信只要工作需要,我能很快速的使用PHP进行开发。}


\section{个人项目}

  \subsection{Python Interpreter}
    \cvline{项目地址:}{\Colorhref{https://github.com/yuyang0/interp-py}{https://github.com/yuyang0/interp-py}}
    \cvline{简介:}{实现了python语言的绝大多数核心功能,包括基本类型,内置操作符,高阶函数,
        OOP,模块等等,实现该解释器的过程让我对python语言的语法与语义有了很深刻的理解(包括python比较差劲的部分)。}
  \subsection{Lisp Interperter}
    \cvline{项目地址:}{\Colorhref{https://github.com/yuyang0/interp-yin}{https://github.com/yuyang0/interp-yin}}
    \cvline{简介:}{一个python写成的类lisp语言的解释器,包含了完整的词法分析,语法分析}

  \subsection{Django Blog}
  \cvline{项目地址:}{\Colorhref{https://github.com/yuyang0/yyblog}{https://github.com/yuyang0/yyblog}}
  \cvline{简介:}{学习django时开发的一个单用户博客,比较简单,但是对我熟悉django框架有很大的帮助。}

\end{document}

